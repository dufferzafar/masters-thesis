
\newpage
\section{Problems with PoW Blockchains} \label{intro-problems}

\fixme{Major portions of this section have been copied from the CCS paper, rewrite?!}

As explained before, in a Proof-of-Work blockchain, every miner tries to solve a computational puzzle and the first one to do so gets to broadcast their block to the network, which the other miners accept and the block is added to the chain. 
Each block adds some weight to the chain depending on the difficulty of the puzzle that they solved. 
This strengthens the chain against future attacks.
In cases where more than one miner solves the puzzle at the same time, a fork is created. 
When other miners observe these competing chains they extend the one with the most cumulative weight (also called the Heaviest Chain Rule). 

Previous work \cite{bitcoinOriginal, Rosenfeld} has modeled the PoW consensus scheme as a Poisson process, which means that blocks get added to the blockchain at random time instants and the weight of the chain increase unsteadily. 
This causes various security and performance problems.

\textbf{Forks} - A fork is created when more than one miner create blocks pointing to the same parent. 
This can happen because they both ended up solving the puzzle simulataneously or perhaps one of the node was lagging behind the network and was not aware of the latest blocks, so ended up mining on an older block. 
These "natural" forks are distinct from forks that might be intentionally created by a malicious attacker.
A fork is resolved when one of the branches of the chain gets longer (acquires more weight) than the others.
% The shorter chain (and all blocks in it) is  
As the majority of the network switches to heavier chain, the blocks in the shorter become \textit{stale} and are essentially discarded. 
Miners who created the stale blocks lose their mining fees, which means their work was "wasted".
During a fork there is uncertainty in the ordering of the blocks in the chain as the network hasn't agreed on a single chain.
% This delay in consensus 
When forks take a long time to get resolved, the blockchain is more prone to attacks such as double-spending and selfish-mining \cite{selfishmining, selfishMiningStrategies}. 

A successful \textbf{double-spending} attack allows an attacker to spend the same bitcoin twice.
% To execute this, the attacker spends some bitcoin that they have on the chain, and then create a fork in the chain
To execute this, the attacker creates two conflicting transactions that both spend the same bitcoin and broadcasts it onto the network. 
To guard against these attacks, users are suggested to wait for multiple blocks to arrive on the chain before accepting a payment or releasing the goods. This duration is known as the \textit{confirmation time}.

\textbf{Selfish-mining}, on the other hand, is a block-withholding strategy where an attacker secretly mines a chain of blocks and does not broadcast it to the network. By suppressing this information the attacker tricks honest nodes into “wasting” their computational power by mining on blocks that already have successors (which are known only to the attacker). 
Once the attacker's chain gets longer than the honest chain, blocks generated by the attacker become part of the eventually agreed-upon heaviest chain, while blocks generated by honest nodes are discarded.
This attack is possible because an attacker with less mining power than the honest miners can occasionally generate a longer private chain.

One way to prevent the occurrence of natural forks is by increasing the block inter-arrival time as that gives the block a longer duration to reach all nodes in the network. 
In Bitcoin, the average time between generation of two blocks is kept to be 10 minutes to prevent natural forks from occurring;
but this also leads to \textbf{reduced throughput} and \textbf{increased confirmation times}.

Throughput of a Blockchain may be measured as the number of transactions that get on to the chain per unit time (transactions per second). 
This depends on two main factors, the block size and the block inter-arrival time. 
Bitcoin has an average block size of around 1 MB while the inter-arrival time is 10 minutes which translates to around 7 to 10 transactions per second.
This is orders of magnitudes less compared to traditional payment networks. 
For example, PayPal is capable of handling a few hundred transactions per second 
whereas VISA can process up to several thousand transactions per second. \cite{scalingBlockchains, visanet}

To prevent double-spending attacks, Bitcoin users are suggested to wait till their transaction is atleast 6 blocks deep in the chain before releasing any goods in lieu of the transaction.
This rule of thumb ensures that the chance of an attacker being able to overtake the main chain is less than a security 0.001\%. This analysis assumes that the attacker has less than 10\% of total hashing power \cite{bitcoinOriginal, Rosenfeld}.
The confirmation time using this rule comes out to one hour, which is impractical for typical scenarios that require instant payment like buying a cup of coffee.

%----------------------------------------------------------------------------------------
